\documentclass{article}
    % General document formatting
    \usepackage[margin=1in]{geometry}
    \usepackage[parfill]{parskip}
    \usepackage[utf8]{inputenc}
    \usepackage{amsmath,amssymb,amsfonts,amsthm}
    \usepackage{url}
    \usepackage{bm}
    \usepackage{graphicx}

\begin{document}

\title{A simple braching process model for Delta-variant and non-Delta-variant incidence of COVID-19 in Germany}
\author{Johannes Bracher}


\maketitle
\bigskip

\textit{This is just a sketch shared via a GitHub repository to facilitate discussion. While I think that the results are reasonable, please keep in mind that this is a rather simple model taking into account only the fact that there are two co-corculating variants. Vaccination, population behaviour, NPIs etc are ignored. Also, I set this up quite quickly and cannot guarantee that there are no technical mistakes.}

\section{Data and notation}

The model involves the following quantities:

\begin{itemize}
\item To describe the epidemic process:
\begin{itemize}
\item $X_t$: The total number of notified cases in week $t$, either from RKI or JHU (both by reporting date to the natinal authorities). Weeks run from Sunday through Saturday to ensure comparability to the Forecast Hub.
\item $X^\Delta_t, X^o_t$: The number of Delta and other cases among the notified cases, respectively (unobserved)
\item $r^\Delta_t, r^o_t$: One plus the weekly growth rate of the Delta variant and other variants combined, respectively.
\item $\psi^\Delta, \psi^o$: Overdispersion parameters for the two variants.
\item $\omega^\Delta_t, \omega^o_t$ auxiliary variables needed to express the negative binomial distributions as a Poisson-gamma mixture.
\item $\lambda^\Delta_t, \lambda^o_t$: Conditional expectations of $X^\Delta_t, X^o_t$ given $X^\Delta_{t - 1}, X^o_{t - 1}$ and $\omega^\Delta_t, \omega^o_t$.
\end{itemize}
\item To describe the sequencing:
\begin{itemize}
\item $N_t$: The number of sequenced cases in the random sample from RKI in week $t$. Note: These weeks are defined as Monday through Sunday.
\item $Y^\Delta_t$: The number among these which are from the Delta variant in week $t$.
\end{itemize}
\end{itemize}

\textbf{Caveat:} The definition of the weeks for the different data streams is not identical, and I am not sure whether one needs to shift one of them slightly to have them actually refer to the same time periods.

\section{Model formulation}

The total number of observed cases is decomposed into
$$
X_t = X^\Delta_t + X^o_t
$$
and assume negative binomial branching processes for $X^\Delta_t$ and $X^o_t$:
\begin{align*}
X^{(\cdot)}_t \ \mid \ \lambda^{(\cdot)}_t & \sim \text{Pois}(\lambda_t)\\
\lambda^{(\cdot)} & = \omega_t \times r_t \times X^{(\cdot)}_{t - 1}
\end{align*}
Here, $\omega^{(\cdot)}_t$ is an auxiliary gamma-distributed random variable,
$$
\omega^{(\cdot)}_t \sim \text{Gamma}(\psi^{(\cdot)}, \psi^{(\cdot)})
$$
meaning that without conditioning on $\omega^{(\cdot)}_t$ the $X^{(\cdot)}_t$ are negative binomial with mean $r^{(\cdot)}_t X^{(\cdot)}_{t - 1}$ and dispersion parameter $\psi^{(\cdot)}$. For the time-varying weekly multiplication factors $r^{(\cdot)}_t$ (one plus growth rate) we assume a random walk structure,
\begin{align*}
r^{(\cdot)}_t & = r^{(\cdot)}_{t - 1} \times \exp(\epsilon^{(\cdot)}_{t})\\
\epsilon^{(\cdot)} & \sim \text{N}(0, \sigma^{2(\cdot)}).
\end{align*}
We thus assume that the $r^{(\cdot)}_t$ show some smoothness and cannot change too quickly. \textbf{Caveat:} This formulation of the random walk is advantageous as it ensures that $r^{(\cdot)}_t$ is always positive. However, it may introduce a slight upward trend on the scale of $r^{(\cdot)}_{t}$ (as $\mathbb{E}(r^{(\cdot)}_{t + 1} \ \mid \ r^{(\cdot)}_{t}) > r^{(\cdot)}_{t}$). However, I am unsure how this translates to the case scale.

For the sequencing results we add the following observation process:
$$
Y^\Delta_t \ \mid \ N_t \sim \text{Bin}(N_t, p^\Delta_t),
$$
where
$$
p^\Delta_t = \frac{X^\Delta_t}{X_t}.
$$

To complete the model we need to specify some prior distributions for the parameters and initialization of the epidemic process:
\begin{align*}
1/\sigma^{2(\cdot)} & \sim \text{Unif}(5, 200)\\
\psi^{(\cdot)} & \sim \text{Unif}(0, 100)\\
X^\Delta_{16} & \sim \text{Unif}(0, 1000.)\\
\end{align*}
These are intended to be ``uninformative'', motivating priors for these parameters is not straightforward.

\section{Results}

I ran the model on the RKI and JHU data and obtained the results shown in Figures 1 and 2. While the data imply slightly different current case levels, in both cases the extrapolation of trends implies a rather sudden flattening of the case counts which had previously been decaying steadily. The data which have accumulated since the last used data point point into this direction, too, even though the flattening may not be as abrupt as implied by the point predictions. Note that the model implies a lot of uncertainty and for more than one week ahead actually makes very uninformative predicitons. In three weeks the Delta variant will have become strongly dominant.

\begin{figure}
\center
\includegraphics[scale=0.6]{plots/plot_RKI_2021-07-01.pdf}
\caption{Results based on RKI data, 1 July 2021.}
\end{figure}

\begin{figure}
\center
\includegraphics[scale=0.6]{plots/plot_JHU_2021-07-01.pdf}
\caption{Results based on JHU data, 1 July 2021.}
\end{figure}
\end{document}
